\section{Running RAVEN}
\label{HowToRun}

% I don't think this is mentioned earlier? Andrea answers :D It mentioned in the Introduction
%As already mentioned,
The RAVEN code is a blend of C++, C, and Python software. The entry point
resides on the Python side and is accessible via a command line interface.
%
After following the instructions in the previous Section, RAVEN is ready to be
used.
%
The \texttt{raven\_framework} script is in the raven folder.
%
To run RAVEN, open a terminal and use the following command (replace \texttt{<inputFileName.xml>} with your RAVEN input file):

\begin{itemize}

  \item \textbf{Any unix-based systems (e.g. Macintosh, Linux, etc.)}:
\begin{lstlisting}[language=bash]
raven_framework <inputFileName.xml>
\end{lstlisting}
  \item \textbf{Windows}:
  \begin{lstlisting}[language=bash]
bash.exe raven_framework <inputFileName.xml>
\end{lstlisting}
  
\end{itemize}

Using \texttt{raven\_framework} is the recommended way to run RAVEN.  In the event bypassing the typical
environment loading and checks is desired, it can also be run via
the \texttt{raven\_framework.py} script using python, with the input file as argument.  However, this is not
recommended, as it will use whatever default versions of Python and other libraries are discovered, rather
than the matching libraries set up during installation.

\nb For Windows systems, we provided a convenient Batch script ( \texttt{raven\_framework.bat} ) for running RAVEN 
avoiding to interact with the Windows command line terminal. More info on how to use it can be found in the RAVEN
\wiki , section \textit{Running RAVEN} (\url{https://github.com/idaholab/raven/wiki/runningRAVEN}).

