\subsubsection{SafestPoint}
\label{SafestPoint}
The \textbf{SafestPoint} PostProcessor provides the coordinates of the farthest
point from the limit surface that is given as an input.
%
The safest point coordinates are expected values of the coordinates of the
farthest points from the limit surface in the space of the ``controllable''
variables based on the probability distributions of the ``non-controllable''
variables.

The term ``controllable'' identifies those variables that are under control
during the system operation, while the ``non-controllable'' variables are
stochastic parameters affecting the system behavior randomly.

The ``SafestPoint'' post-processor requires the set of points belonging to the
limit surface, which must be given as an input.
%
The probability distributions as ``Assembler Objects'' are required in the
``Distribution'' section for both ``controllable'' and ``non-controllable''
variables.

The sampling method used by the ``SafestPoint'' is a ``value'' or ``CDF'' grid.
%
At present only the ``equal'' grid type is available.

\ppType{Safest Point}{SafestPoint}

\begin{itemize}
  \item \xmlNode{Distribution}, \xmlDesc{Required}, represents the probability
  distributions of the ``controllable'' and ``non-controllable'' variables.
  %
  These are \textbf{Assembler Objects}, each of these nodes must contain 2
  attributes that are used to identify those within the simulation framework:
        \begin{itemize}
    \item \xmlAttr{class}, \xmlDesc{required string attribute}, is the main
    ``class'' the listed object is from.
                \item \xmlAttr{type}, \xmlDesc{required string attribute}, is the object
    identifier or sub-type.
        \end{itemize}
             \item  \xmlNode{outputName}, \xmlDesc{string, required field}, specifies the name of the output variable where the probability is going to be stored.
               \nb This variable name must be listed in the \xmlNode{Output} field of the Output DataObject
        \item \xmlNode{controllable}, \xmlDesc{XML node, required field},  lists the controllable variables.
  %
  Each variable is associated with its name and the two items below:
        \begin{itemize}
                \item \xmlNode{distribution} names the probability distribution associated
    with the controllable variable.
    %
                \item \xmlNode{grid} specifies the \xmlAttr{type}, \xmlAttr{steps}, and
    tolerance of the sampling grid.
    %
        \end{itemize}
        \item \xmlNode{non-controllable}, \xmlDesc{XML node, required field}, lists the non-controllable variables.
  %
  Each variable is associated with its name and the two items below:
        \begin{itemize}
                \item \xmlNode{distribution} names the probability distribution associated
    with the non-controllable variable.
    %
                \item \xmlNode{grid} specifies the \xmlAttr{type}, \xmlAttr{steps}, and
    tolerance of the sampling grid.
    %
                \end{itemize}
\end{itemize}

\textbf{Example:}
\begin{lstlisting}[style=XML,morekeywords={name,subType,class,type,steps}]
<Simulation>
  ...
    <Models>
    ...
    <PostProcessor name='SP' subType='SafestPoint'>
      <Distribution  class='Distributions'  type='Normal'>x1_dst</Distribution>
      <Distribution  class='Distributions'  type='Normal'>x2_dst</Distribution>
      <Distribution  class='Distributions'  type='Normal'>gammay_dst</Distribution>
      <controllable>
        <variable name='x1'>
          <distribution>x1_dst</distribution>
          <grid type='value' steps='20'>1</grid>
        </variable>
        <variable name='x2'>
          <distribution>x2_dst</distribution>
          <grid type='value' steps='20'>1</grid>
        </variable>
      </controllable>
      <non-controllable>
        <variable name='gammay'>
          <distribution>gammay_dst</distribution>
          <grid type='value' steps='20'>2</grid>
        </variable>
      </non-controllable>
    </PostProcessor>
    ...
  </Models>
  ...
</Simulation>
\end{lstlisting}
