%%%%%%%%%%%%%%%%%%%%%%%%%%%%%%%%%%%%%
%%%%%% OPENMODELICA INTERFACE  %%%%%%
%%%%%%%%%%%%%%%%%%%%%%%%%%%%%%%%%%%%%
\subsection{OpenModelica Interface}
OpenModelica (\url{http://www.openmodelica.org}) is an open source implementation of the Modelica simulation language.  Modelica is "a non-proprietary,
object-oriented, equation based language to conveniently model complex physical systems containing, e.g., mechanical, electrical, electronic, hydraulic,
thermal, control, electric power or process-oriented subcomponents."\footnote{\url{http://www.modelica.org}}.  Modelica models are specified in text files
with a file extension of .mo.  A standard Modelica example called BouncingBall which simulates the trajectory of an object falling in one dimension from a
height is shown as an example:
\begin{lstlisting}
model BouncingBall
  parameter Real e=0.7 "coefficient of restitution";
  parameter Real g=9.81 "gravity acceleration";
  Real h(start=1) "height of ball";
  Real v "velocity of ball";
  Boolean flying(start=true) "true, if ball is flying";
  Boolean impact;
  Real v_new;
  Integer foo;

equation
  impact = h <= 0.0;
  foo = if impact then 1 else 2;
  der(v) = if flying then -g else 0;
  der(h) = v;

  when {h <= 0.0 and v <= 0.0,impact} then
    v_new = if edge(impact) then -e*pre(v) else 0;
    flying = v_new > 0;
    reinit(v, v_new);
  end when;

end BouncingBall;
\end{lstlisting}

\subsubsection{Files}
An OpenModelica installation specific to the operating system is used to create a stand-alone executable program that performs the model calculations.
A separate XML file containing model parameters and initial conditions is also generated as part of the build process.  The RAVEN OpenModelica interface
modifies input parameters by changing copies of this file.  Both the executable and XML parameter file names must be provided to RAVEN.  In the case of
the BouncingBall model previously mentioned on the Windows operating system, the \textless Files\textgreater  specification would look like:
\begin{lstlisting}[style=XML]
<Files>
  <Input name='BouncingBall_init.xml' type=''>BouncingBall_init.xml</Input>
  <Input name='BouncingBall.exe' type=''>BouncingBall.exe</Input>
</Files>
\end{lstlisting}
\subsubsection{Models}
OpenModelica models may provide simulation output in a number of formats.  The particular format used is specified during the model generation
process.  RAVEN works best with Comma-Separated Value (CSV) files, which is one of the possible output format options.  Models are generated
using the OpenModelica Shell (OMS) command-line interface, which is part of the OpenModelica installation.  To generate an executable that provides
CSV-formatted output, use OMSl commands as follows:
\lstset{
    frame=single,
    breaklines=true,
    postbreak=\raisebox{0ex}[0ex][0ex]{\ensuremath{\color{red}\hookrightarrow\space}}
}
 \begin{enumerate}
\item Change to the directory containing the .mo file to generate an executable for:
\begin{lstlisting}
>> cd("C:/MinGW/msys/1.0/home/bobk/projects/raven/framework/CodeInterfaces/OpenModelica")
"C:/MinGW/msys/1.0/home/bobk/projects/raven/framework/CodeInterfaces/OpenModelica"
\end{lstlisting}
\item Load the model file into memory:
\begin{lstlisting}
>> loadFile("BouncingBall.mo")
true
\end{lstlisting}
\item Create the model executable, specifying CSV output format:
\begin{lstlisting}
>> buildModel(BouncingBall, outputFormat="csv")
{"C:/MinGW/msys/1.0/home/bobk/projects/raven/framework/CodeInterfaces/OpenModelica/BouncingBall","BouncingBall_init.xml"}
Warning: The initial conditions are not fully specified. Use +d=initialization for more information.
\end{lstlisting}
At this point the model executable and XML initialization file should have been created in the same directory as the original model file.
\end{enumerate}
The model executable is specified to RAVEN using the \textless Models\textgreater  section of the input file as follows:
\begin{lstlisting}[style=XML]
<Simulation>
    ...
  <Models>
    <Code name="BouncingBall" subType = "OpenModelica">
      <executable>BouncingBall.exe</executable>
    </Code>
  </Models>
    ...
</Simulation>
\end{lstlisting}
\subsubsection{CSV Output}
The CSV files produced by OpenModelica model executables require adjustment before it may be read by RAVEN.
The first few lines of original CSV output from the
BouncingBall example is shown below:
\begin{lstlisting}
"time","h","v","der(h)","der(v)","v_new","foo","flying","impact",
0,1,0,0,-9.810000000000001,0,2,1,0,
  ...
\end{lstlisting}
RAVEN will not properly read this file as-generated for two reasons:
\begin{itemize}
  \item The variable names in the first line are each enclosed in double-quotes.
  \item Each line has a trailing comma.
\end{itemize}
 The OpenModelica inteface will automatically remove the double-quotes and trailing commas through its implementation of the
finalizeCodeOutput function.
