\section{Reduced Order Modeling through RAVEN}
\label{sec:ROMraven}
The development of high-fidelity codes, for thermal-hydraulic systems
and integrated multi-physics, has undergone a significant acceleration
in the last years. Multi-physics codes simulate
multiple physical models or multiple simultaneous physical phenomena,
in a integrated solving environment. Multi-physics typically
solves coupled systems of partial differential equations, generally
characterized by several different geometrical and time scales.

The new multi-physics codes are characterized by remarkable
improvements
in the approximation of physics (high approximation order and reduced
use of empirical correlations). This greater fidelity is generally
accompanied by a greater computational effort (increased calculation time). This peculiarity is an
obstacle in the application of  computational techniques of
quantification of uncertainty and risk associated with the operation of
particular industrial plant (e.g., a nuclear reactor).

A solution to this problem is represented by the
usage
of highly effective sampling strategies. Sometimes also these
approaches is not enough
in order to perform a comprehensive UQ and PRA analysis. In these
cases the help of reduced order modeling is essential.

RAVEN has support of several different ROMs,
such as:
\begin{enumerate}
  \item \textit{Nearest Neighbors approaches}
  \item \textit{Support Vector Machines}
  \item \textit{Inverse Weight regressors}
  \item \textit{Spline regressors }, etc.
\end{enumerate}

A ROM, also known a surrogate
model, is a mathematical representation of a system, used to predict
a FOM of a physical system.

The ``training'' is a process of setting the internal parameters of the ROM from a set
of samples generated the physical model, i.e.,
 the high-fidelity simulator (RELAP-7, RELAP5
3D, PHISICS, etc.),
\begin{figure}[h!]
  \centering
  \includegraphics[width=1.0\textwidth]  {pics/ROMexampleOfPhysicalSystem.png}
  \caption{Example of reduced order model representation of physical system (regression).}
  \label{fig:ROMexampleOfPhysicalSystem}
\end{figure}

Two characteristics of these models
are generally assumed (even if exceptions are possible):
\begin{enumerate}
  \item The higher the number of realizations in the training sets, the
higher is the accuracy of the prediction performed by the ROM is. This
statement is true for most of the cases, although some ROMs might be
subject to the over-fitting issues. The over-fitting phenomenon is not
analyzed here, since its occurrence highly depends on the
algorithm type, and, hence, the problem needs to be analyzed for all
the large number of ROM types available
  \item The smaller the size of the input (uncertain) domain with
  respect to the variability of the system response, the more likely the
  ROM is able to represent the system response space.
\end{enumerate}

The goals of this section are about learning how to:
 \begin{enumerate}
   \item Set up a sampling strategy to construct multiple ROMs, perturbing a driven code
   \item Train the different ROMs with the data-set obtained by the applied sampling strategy;
   \item Use the same sampling strategy, perturbing the ROMs
   \item Plot the responses of the driven code and ROMs, respectively.
\end{enumerate}
In order to accomplish these tasks, the following RAVEN \textbf{Entities} (XML blocks in the input files) need to be defined:
\begin{enumerate}
   \item \textbf{\textit{RunInfo}}:
     \xmlExample{framework/user_guide/ReducedOrderModeling/reducedOrderModeling.xml}{RunInfo}
   As in the other examples, the \textit{RunInfo} \textbf{Entity} is intended  to set up the analysis sequence that
   needs to be performed. The number of steps specified in (\xmlNode{Sequence}) are sequentially run, eight steps in this specific case, using the number of processors assigned in (\xmlNode{batchSize}).
   \\In the first step, the model is going to be sampled. The obtained results are going to be used to  train three different ROMs.These ROMs are sampled by the same strategy used in the first step in order to compare the ROMs' responses with the ones coming from the original model.
   \item \textbf{\textit{Models}}:
     \xmlExample{framework/user_guide/ReducedOrderModeling/reducedOrderModeling.xml}{Models}
 As mentioned earlier, the goal of this example is the employment of
 a sampling strategy in order to construct multiple types of ROMs.
 \\Indeed, in addition to an External model,
 three different ROMs (GP, SVM and IDW) are here specified. The ROMs will be
 constructed (``trained'') through the data-set generated by the sampling of the External model. Once trained, they are going  to be used in place of the original model.
 \\As it can be seen, the ROMs will be constructed considering two features ($v0,\, and angle,\,$) and two targets  ($r \, and \, t$).
   \item \textbf{\textit{Distributions}}:
     \xmlExample{framework/user_guide/ReducedOrderModeling/reducedOrderModeling.xml}{Distributions}
  In the Distributions XML section, the stochastic model for the
  uncertainties are reported. In
  this case two distributions are defined:
  \begin{itemize}
    \item $vel\_dist \sim \mathbb{N}(30,5)$, used to model the uncertainties
    associated with  the \textit{velocity};
    \item  $angle\_dist \sim \mathbb{U}(5,85)$,  used to
    model the uncertainties associated with the \textit{angle}.
  \end{itemize}
   \item \textbf{\textit{Samplers}}:
     \xmlExample{framework/user_guide/ReducedOrderModeling/reducedOrderModeling.xml}{Samplers}
  To obtain the data-set on which the data mining algorithms are going to be applied, a \textit{MonteCarlo} sampling approach is employed here.
   \item \textbf{\textit{DataObjects}}:
     \xmlExample{framework/user_guide/ReducedOrderModeling/reducedOrderModeling.xml}{DataObjects}
  In this block, six \textit{DataObjects} are defined: 1) PointSet
  named ``samples'' used to collect the final outcomes of the code, 2)
  HistorySet named ``histories'' in which the full time responses of the
  variables are going to be stored, 3) PointSet named
  ``inputPlaceHolder'' used in the \textit{role} of \xmlNode{Input} for the ROMs sampling;
  4) PointSet named ``samplesGP'' used to collect the final outcomes (sampling) of the Gaussian Process (GP) ROM;
  5) PointSet named ``samplesInverse'' used to collect the final outcomes (sampling) of the Inverse Distance Weighting (IDW) ROM;
  6) PointSet named ``samplesSVM'' used to collect the final outcomes (sampling) of the Support Vector Machine (SVM) ROM.
 %%%%%%%%%%%%%%%%%%%%%%%%%%%%%%%%%%%%%%%%%%%%%%%%%%%%%%%%%%
 %%%%%%%%%%%%%%%%%%%%%%%%%%%%%%%%%%%%%%%%%%%%%%%%%%%%%%%%%%
 %figure samples
 \begin{figure}[h!]
  \centering
  \includegraphics[scale=0.7]{../../tests/framework/user_guide/ReducedOrderModeling/gold/ROMConstruction/1-samplesPlot3D_scatter-scatter.png}
  \caption{Plot of the samples generated by the Monte Carlo sampling}
  \label{fig:ROMgrid_pointsets}
 \end{figure}
 %%%%%%%%%%%%%%%%%%%%%%%%%%%%%%%%%%%%%%%%%%%%%%%%%%%%%%%%%%
 %%%%%%%%%%%%%%%%%%%%%%%%%%%%%%%%%%%%%%%%%%%%%%%%%%%%%%%%%%
   \item \textbf{\textit{OutStreams}}:
     \xmlExample{framework/user_guide/ReducedOrderModeling/reducedOrderModeling.xml}{OutStreams}
     This model makes use of two Print OutStreams and five Plot OutStreams:
     \begin{itemize}
       \item ``samples,'' which writes the contents of the point-wise training samples to CSV,
       \item ``histories,'' which writes the contents of the history-wise training samples to linked CSVs,
       \item ``historyPlot,'' which plots the evolution of the training samples,
       \item ``samplesPlot3D,'' which plots the final state of the training samples with relation to the
         outputs of interest,
       \item ``samplesPlot3DROMgp,'' which plots the validation samples of the Gaussian Process ROM,
       \item ``samplesPlot3DROMsvm,'' which plots the validation samples of the Support-Vector Machine ROM,
       \item ``samplesPlot3Dinverse,'' which plots the validation samples of the multidimensional Inverse
         Weight ROM.
     \end{itemize}
     The 3D plots of the samples as well as the ROM samples can be used as a view-norm validation of the ROMs.
   \item \textbf{\textit{Steps}}:
     \xmlExample{framework/user_guide/ReducedOrderModeling/reducedOrderModeling.xml}{Steps}
  %%%%%%%%%%%%%%%%%%%%%%%%%%%%%%%%%%%%%%%%%%%%%%%%%%%%%%%%%%
 %figure samples
 \begin{figure}[h!]
  \centering
  \includegraphics[scale=0.7]{../../tests/framework/user_guide/ReducedOrderModeling/gold/ROMConstruction/1-historyPlot_scatter-scatter.png}
  \caption{Plot of the histories generated by the Monte Carlo method}
  \label{fig:ROMgrid_histories}
 \end{figure}
   %%%%%%%%%%%%%%%%%%%%%%%%%%%%%%%%%%%%%%%%%%%%%%%%%%%%%%%%%%
 %figure samples
 \begin{figure}[h!]
  \centering
  \includegraphics[scale=0.7]{../../tests/framework/user_guide/ReducedOrderModeling/gold/ROMConstruction/1-samplesPlot3DROMgp_scatter-scatter.png}
  \caption{Plot of the samples generated by the Monte Carlo sampling applied on the Gaussian Process ROM}
  \label{fig:ROMgp_samples}
 \end{figure}
 %%%%%%%%%%%%%%%%%%%%%%%%%%%%%%%%%%%%%%%%%%%%%%%%%%%%%%%%%%
 %%%%%%%%%%%%%%%%%%%%%%%%%%%%%%%%%%%%%%%%%%%%%%%%%%%%%%%%%%
   Finally, all the previously defined \textbf{Entities} can be combined in
   the \xmlNode{Steps} block. As inferable,
   eight \xmlNode{Steps} have been inputted:
   \begin{itemize}
     \item \xmlNode{MultiRun} named ``sample'', used to run the multiple
     instances of the driven code and
     collect the outputs in the two \textit{DataObjects}. As it can be
     seen, the \xmlNode{Sampler} is inputted to communicate to the
     \textit{Step} that the driven code needs to
     be perturbed through the Grid sampling strategy;
     \item \xmlNode{RomTrainer} named ``trainROMGaussianProcess'', used to construct (``train'')
     the GP ROM, based on the data-set generated in the  ``sample'' \textbf{Step};
     \item \xmlNode{RomTrainer} named ``trainROMsvm'', used to construct (``train'')
     the SVM ROM, based on the data-set generated in the  ``sample'' \textbf{Step};
     \item \xmlNode{RomTrainer} named ``trainROMinverse'', used to construct (``train'')
     the IDW ROM, based on the data-set generated in the  ``sample'' \textbf{Step};
     \item \xmlNode{MultiRun} named ``sampleROMGaussianProcess'', used to run the multiple
     instances of the previously constructed GP ROM and
     collect the outputs in the PointSet \textit{DataObject}. As it can be
     seen, the same \xmlNode{Sampler} used for perturbing the original model is here used.
     \item \xmlNode{MultiRun} named ``sampleROMsvm'', used to run the multiple
     instances of the previously constructed Support Vector Machine ROM and
     collect the outputs in the PointSet \textit{DataObject}. As it can be
     seen, the same \xmlNode{Sampler} used for perturbing the original model is here used.
     \item \xmlNode{MultiRun} named ``sampleROMInverse'', used to run the multiple
     instances of the previously constructed Inverse Distance Weight ROM and
     collect the outputs in the PointSet \textit{DataObject}. As it can be
     seen, the same \xmlNode{Sampler} used for perturbing the original model is here used.
     \item  \xmlNode{IOStep} named ``writeHistories'', used to 1) export
     the ``histories'' and ``samples''  \textit{DataObjects}
     \textbf{Entity} in a CSV file and 2) plot the responses of the sampling performed on the physical model, GP ROM,
     SVM ROM and IDW ROM in  PNG files and on the screen.
   \end{itemize}
\end{enumerate}

  %figure samples
 \begin{figure}[h!]
  \centering
  \includegraphics[scale=0.7]{../../tests/framework/user_guide/ReducedOrderModeling/gold/ROMConstruction/1-samplesPlot3DROMsvm_scatter-scatter.png}
  \caption{Plot of the samples generated by the Monte Carlo sampling applied on the Support Vector Machine ROM}
  \label{fig:ROMsvm_samples}
 \end{figure}
 %%%%%%%%%%%%%%%%%%%%%%%%%%%%%%%%%%%%%%%%%%%%%%%%%%%%%%%%%%
  %%%%%%%%%%%%%%%%%%%%%%%%%%%%%%%%%%%%%%%%%%%%%%%%%%%%%%%%%%
  %figure samples
 \begin{figure}[h!]
  \centering
  \includegraphics[scale=0.7]{../../tests/framework/user_guide/ReducedOrderModeling/gold/ROMConstruction/1-samplesPlot3DROMinverse_scatter-scatter.png}
  \caption{Plot of the samples generated by the Monte Carlo sampling applied on the Inverse Distance Weight ROM}
  \label{fig:ROMinverse_samples}
 \end{figure}
 %%%%%%%%%%%%%%%%%%%%%%%%%%%%%%%%%%%%%%%%%%%%%%%%%%%%%%%%%%
 Figure \ref{fig:ROMgrid_histories}
 shows the range $r$ for different velocity and angle.
 Figure \ref{fig:ROMgrid_pointsets} shows the final responses  of the sampling employed using the driven code.

Figures \ref{fig:ROMgp_samples}, \ref{fig:ROMsvm_samples} and \ref{fig:ROMinverse_samples}  show the final responses  of the sampling employed using the Gaussian Process, Support Vector Machines and Inverse Distance Weight ROMs, respectively.
It can be clearly noticed that the responses of the ROMs perfectly match the outcomes coming from the original model (see Figure   \ref{fig:ROMgrid_pointsets}).








